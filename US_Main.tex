\documentclass{article}

% Language setting
% Replace `english' with e.g. `spanish' to change the document language
\usepackage[english]{babel}

% Set page size and margins
% Replace `letterpaper' with`a4paper' for UK/EU standard size
\usepackage[letterpaper,top=2cm,bottom=2cm,left=3cm,right=3cm,marginparwidth=1.75cm]{geometry}

% Useful packages
\usepackage{amsmath}
\usepackage{graphicx}
\usepackage[colorlinks=true, allcolors=blue]{hyperref}

\title{\textbf{Assessment\\CASA0002 Urban Simulation\\MSc Urban Spatial Science}}
\author{\textbf{Andrés Restrepo Jiménez}}



\usepackage{natbib}
\bibliographystyle{abbrvnat}
\hypersetup{citecolor=black}
\setcitestyle{authoryear,open={(},close={)}}
\setcounter{secnumdepth}{4}

\begin{document}
\maketitle

% \begin{abstract}
% Your abstract.
% \end{abstract}

\href{https://github.com/rpoandres/US_Assessment.git}{\centerline{GitHub Repo}} 

\section{London’s underground resilience}

% Your introduction goes here! Simply start writing your document and use the Recompile button to view the updated PDF preview. Examples of commonly used commands and features are listed below, to help you get started.

% Once you're familiar with the editor, you can find various project setting in the Overleaf menu, accessed via the button in the very top left of the editor. To view tutorials, user guides, and further documentation, please visit our \href{https://www.overleaf.com/learn}{help library}, or head to our plans page to \href{https://www.overleaf.com/user/subscription/plans}{choose your plan}.

\subsection{Topological network}

\subsubsection{Centrality measures}

\paragraph{Degree centrality}\mbox{}\\

The degree of a node is the number of edges that node is connected with \citep{arcauteNetworksUrbanSimulation2023}. Hence, the degree centrality measure account for the number of edges a node is connected to. The higher the value, the higher number of edges (links) are connected to that node. Although simple, the degree centrality comes as a valuable reference when assessing networks \citep{newmanNetworks2018}. The degree centrality measure $d$ for node $i$ is defined as 

\[D_{i} = \frac{v}{n-1}\]

where $v$ is the number of edges connected to node $i$ and $n$ is number of nodes in the network. It represents the faction of nodes that node $i$ is connected to, as $n-1$ is the number of possible connections of node $i$ in the network \citep{hagbergExploringNetworkStructure2008}.

In the underground network context, the degree centrality of a station would measure the number of connections of a stations to other stations in the underground. Thus, this measure would detect the stations with the higher level of connectivity in the underground when just considering the number of direct connections to other stations. If the station with the highest degree centrality measure value is removed, it will caused the biggest effect in the network in terms of number of disrupted immediate connections.


\paragraph{Betweenness centrality}\mbox{}\\

The betweenness centrality measure calculates the number of shortest paths between all possible node pairs in a network that go though a node \citep{arcauteNetworksUrbanSimulation2023}.

To calculate this measure, it is required to determine if a given node is the shortest path between every node pair in a network $s^{j}_{k}$ as follows


\[s^{i}_{jk}=\begin{cases}
1, & \text{if node $i$ is included in geodesic path between $j$ to $k$}.\\
0, & \text{if not included}.
  \end{cases}
\]

Next, the betweenness centrality measure is defined as 

\[B_{i}=\sum_{jk} s^{i}_{jk}\]

Specially in the topological case, there is a high chance of having more than one geodesic path between $j$ and $k$. To prevent this, the measure must be normalized by the number of geodesic paths $g$ for node pairings as follows

\[B_{i}=\sum_{jk} \frac {s^{i}_{jk}}{g_{jk}}\]



\citep{newmanNetworks2018} cited by \citep{marinReviewCentralityMeasures2022}

In the case of the underground network, the betweeness centrality identifies the station which the highest number of shortest paths between node pairs go thought it. If removed, all the shortest path that go though that station will recomputed to longer alternatives paths to connect origin and destination stations, negatively affecting the system's efficiency.

\paragraph{Eigenvector centrality}\mbox{}\\

The eigenvector centrality measure takes into account the connections of the connections of the node of interest. Similarly to the degree centrality, the eigen vector is considered an extension of this measure as it also incorporates the number of connections that the immediate connections of a node has \citep{newmanNetworks2018}.


To calculate this measure in node $i$, it is required to consider the adjacency matrix as follows for the second degree connections with nodes $j$

\[x_{i}={k^{-1}}\sum_{j=1}^{n} {A_{ij}}{x{j}}\] where ${A_{ij}}$ is the adjacency matrix.

In the underground network case, the eigenvector centrality measures captures the number of connections of the stations connected to the station of interested. A station can have a low degree centrality on its won, however if one of those connection is with a station highly connected to other multiple stations, the eigenvector centrality will flag this situation.

\paragraph{Centrality measures results}\mbox{}\\

Next, a summary of the centrality measures results is presented:

\begin{figure}[htp]
    \centering
    \includegraphics[width=15cm]{Centrality_Results.jpg}
    \caption{Result of centrality measure in the topological network}
    \label{fig:centrality_results}
\end{figure}

With the centrality measures described above, in the 3 cases, Stratford is the station ranked first. Bank and Monument station, is ranked second also in all measures. And at the third position, Liverpool Street station is ranked second in two of the tree centrality measures.

As there are some stations in the degree centrality that have the same value in the measure, their position in the ranking my differ according to the sorting method between stations.

\subsubsection{Impact measures}

\paragraph{Number of connected components}\mbox{}\\

The number of connected components when removing stations from the network, allows to identify if the removal of a given station using disrupts the network to a point where there is more than one connected component only. As the node removal is carried out, this impact measure will flag when the disruption of the network is strong enough to cause the isolation of one more node. Considering that the node removal will be developed using the different rankings of the respective centrality measures selected, it would be possible to compare the level of disruption between centrality measures.

\paragraph{Size of largest connected component}\mbox{}\\

Similarly to the above mentioned, the size of the remaining connected component is a valuable reference for the damage that the node removal procedure is causing in the network. As there might be different scenarios of disruption in the network with the same number of connected components, the size of the largest connected component plays a complementary role of the number itself of connected components. In that sense, the combination of both impact measure would provide detailed and comparable references when assessing the severity of the damage and the resilience of the underground network.

\subsubsection{Node removal}


\begin{figure}[htp]
    \centering
    \includegraphics[width=15cm]{Results_Impact_Measures_Total.jpg}
    \caption{Result of the impact measures by non-sequential and sequential node removal strategies}
    \label{fig:impact_results}
\end{figure}

In Figure \ref{fig:impact_results}, the results of the impact measure are presented, for both non-sequential and sequential node removal strategies. In each case, the impact measure is shown for each centrality measure used in the impact assessment. 

In the non-sequential strategy, both eigenvector and betweenness centrality shown the same  results in both impact measures. The degree centrality measure captured slightly better the increase in number of connected components, specially after removing the 9th highest ranked node. The opposite happens with the size of the largest connected component, in which the eigenvector and betweenness centrality measures flagged slightly better the damage in the network, although the degree centrality outperforms them after removing 15 nodes.

With the sequential node removal strategy, the degree and the betweenness centrality had very similar results in term of the number of connected components. On the other hand, the eigenvector centrality started close to his peers but after node number 10, it has less effective performance in reflecting the network disturbance. Regarding the size of the largest component in the sequential strategy, the centrality measure that seems to have caused the greater effect is the betweenness centrality, as it cased a more dramatically reduction of the size of the largest component, compared to his peers.

Overall, the measure that seems to reflect more accurately the impact when removing nodes is the betweenness centrality, as it has shown a significantly difference in terms of the size of the largest connected component when following the sequential strategy.

\subsection{Flows: weighted network}

The use of additional information as weight variable when assessing the underground network, can bring insight-full reference apart from just considering strictly the topological network. Ideally, the centrality measures used in the above section, should leverage in this additional information to better assess the strengths and weaknesses from a more realistic perspective. In this case, the flows of passengers will be considered in the impact and resilience assessment as they reflect the system in real-life circumstances.

\subsubsection{Adjusted centrality measures}

In order to reflect the flows of passengers in the underground network, the centrality measures need to have minor adjustments.

In the case of the betweenness centrality, before incorporating the flows in the calculations, they must be inverted, as this measure expects to have a length or distance in this variable. Considering that the flows are expressed in terms of passengers, the highest flows of should be related to the shortest distance between nodes in the calculation. To achieve this, the quantity of flows should be inverted and stored as an edge attribute in the network.

For the eigenvector centrality measure, it not necessary yo invert the flows, is the calculation would interpret a higher flow as a more influential node in the network. For the degree centrality measures, it remain same as before considering that this measure is oriented to track the station with the highest number of immediate connections to other stations.

\begin{figure}[htp]
    \centering
    \includegraphics[width=15cm]{Weighted_Centrality_Results.jpg}
    \caption{Result of adjusted centrality measures}
    \label{fig:weighted_centrality_results}
\end{figure}

The results for the betweenness and eigenvector centrality are different to the results of the topological network. These two measures effectively are using the flows as weights to reflect the flow of passengers in each case. The highest ranked station are Green Park and Waterloo, for the weighted betweeness and the weighted eigenvector centrality, respectively. In the topological network, the highest ranked in both measure was Stratford.

\subsubsection{Adjusted impact measure}

For the assessment of the impact in the network, the total flows of the largest connected component were added to determine how the node removal is affecting the normal flow of passengers in the network. The inclusion of this information, would allow to identify how is the isolation impact when there is disturbance in the network in term of flows removed from the connected network, or in other word, decrease of flows in the largest connected component.

\paragraph{Total flows of the largest component}\mbox{}\\



\subsubsection{Node removal}

\section{Spatial interaction models}


\subsection{Models and calibration}

\subsubsection{Introduction}

The spatial interaction models of interest are derived from the basic gravitational model, which embodies Newton's law of universal gravitation \citep{battyUrbanModellingAlgorithms1976}. Newton's law states that ''...a particle in the universe would interact with other particles with a force directly proportional to the product of their masses and inversely proportional to the between them.'' \citep{battyUrbanSimulationSpatial2023}.

When applying gravitation model in urban and regional context, the attractiveness and the distance between two locations, can be seen respectively as the masses and the distance mentioned in Newton's law. In short, the spatial interaction model leverage on the principles of the gravitational models to predict how locations within a city of a regional system would spatially interact with each other based on the distance and the level of attraction between them.

As the spatial interaction model can vary to reflect the different type that might happened within a system, the family of spatial interaction model was introduced and described next, \citep{wilsonEntropyUrbanRegional2011} and \citep{wilsonFamilySpatialInteraction1971} cited by \citep{battyUrbanModellingAlgorithms1976}:

\paragraph{The unconstrained model}\mbox{}\\

The unconstrained model allows flows within the system to interact freely and apply in full the principles. As the intensity of the attraction increases, the flows will also increase. Inversely, when the distance also noted as the cost of interaction increases, the flows will decrease between the locations of interest. The model can be defined as follows \citep{arcauteSpatialInteractionModelling2023}

\[T_{ij} = k \frac{O_i^\alpha  D_j^\gamma}{ d_{ij}^\beta}\] where 

\begin{itemize}
  \item $T_{ij}$ represents the interaction of flows between the location $i$ (origin) and the location $j$ (destination) within the system. Mathematically,  $T_{ij}$ is a matrix with the $i$ origins as rows and the $j$ destinations as columns.
  \item $O_{i}$ is a vector that contains the values of the attribute that reflect the intensity of future interaction of location $i$ as on origin with the rest of the system.
  \item $D_{j}$ is a vector that contains the values of the attribute that captures how intense would be the interactions of location $j$ as an destination in the system.
    \item $d$ is a matrix that contains the distance of cost of interaction between every origin $i$ and destination $j$ of the system.
    \item $k$, $\alpha$, $\gamma$ and $\beta$ are parameters of the model that need to be estimated.
\end{itemize}

$k$ is the balancing factor or the constant of proportionality, and is defined as

\[k = \frac{T}{\sum_i \sum_j O_i^\alpha  D_j^\gamma  d_{ij}^{-\beta}}\] where $T$ represents the total of observed flows between the $i$ origins and $j$ destinations in the system and is denoted by \[T= \sum_i \sum_j T_{ij}\]
\paragraph{Production-constrained model}\mbox{}\\

The production-constrained model varies from the general gravitational model (unconstrained-model) as the flows from every $i$ origins is given and would be a constrain when predicting interactions with the $j$ destinations of the system. This variation of the general model is commonly used to asses the flows residential locations ($i$ origins) to areas with higher commercial densities or retail use ($j$ destinations) \citep{wilkinsonSpatialInteractionModelling2023}. The model is defined as

\[T_{ij} = A_i O_i D_j^\gamma d_{ij}^{-\beta}\] where the flows from the $i$ origins is fixed and defined by \[O_i = \sum_j T_{ij}\] and $A$ the balancing factor is defined as

\[A_i = \frac{1}{\sum_j D_j^\gamma d_{ij}^{-\beta}}\] similarly to the role that $k$ plays in the unconstrained model, making sure that the total origin flows fits the given constrained.  Regarding the other components of the equation

\begin{itemize}
  \item $T_{ij}$ represents the interaction of flows between the location $i$ (origin) and the location $j$ (destination) within the system. 
  \item $O_{i}$ is the constrained vector equivalent to the total flows from origin $i$ to destinations $j$ in the system.
  \item $D_{j}$ is a vector that contains the values of the attribute that captures how intense would be the interactions of location $j$ as an destination in the system.
    \item $d$ is a matrix that contains the distance of cost of interaction between every origin $i$ and destination $j$ of the system.
    \item $A_{i}$, $\gamma$ and $\beta$ are parameters of the model that need to be estimated.
\end{itemize}

\paragraph{Attraction-constrained model}\mbox{}\\

Similarly to the origin-constrained model, the attraction-constrained model predicts the flows between origin $i$ and destination $j$, by constraining the in-coming flows to the destinations $j$ in the system. Is mainly used to estimate the relocation's patterns within a urban system, when a new source of employment is located in certain location ($j$ destination). As the hypothetical new employer is given in a specific site ($j$) and the new jobs are known or estimated, the origin-constrained model predicts where would people relocate to benefit from the new jobs, considering the  cost of interaction $d$ to access the new job positions created. The model is defined as follows

\[T_{ij} = D_j B_j O_i^\alpha d_{ij}^{-\beta}\] where the flows to destination $j$ are a known constrained and defined by \[D_j = \sum_i T_{ij}\] and $B$ as balancing factor is defined as \[B_j = \frac{1}{\sum_i O_i^\alpha d_{ij}^{-\beta}}\]. The other component of the attraction-constrained model are

\begin{itemize}
  \item $T_{ij}$ represents the interaction of flows between the location $i$ (origin) and the location $j$ (destination) within the system. 
  \item $D_{j}$ is the constrained vector equivalent of all flows going to every $j$ destination $j$ from origin $i$ in the system.
  \item $O_{i}$ is a vector that contains the values of the attribute that captures how intense would be the interactions of location $i$ as origins in the system.
    \item $d$ is a matrix that contains the distance of cost of interaction between every origin $i$ and destination $j$ of the system.
    \item $B_{j}$, $\alpha$  and $\beta$ are parameters of the model that need to be estimated.
\end{itemize}


\paragraph{Doubly constrained model}\mbox{}\\

As oppose to the unconstrained model, the doubly constrained model estimates the flows between origin $i$ and destination $j$ by applying constrains in both the origin $O_{i}$ and destination $D_{j}$ vectors. This type of model is useful to assess inter modal mobility patterns when more than one mode of transportation is available in the system, when assessing mobility patterns overtime and when there is variation in the other parameters like the distance or cost of interaction \citep{wilkinsonSpatialInteractionModelling2023}. The model is defined as

\[T_{ij} = A_i B_j O_i D_j d_{ij}^{-\beta}\] where there are constrains for both origin $O_{i}$ and destination $D_{j}$ vectors as \[O_i = \sum_j T_{ij}\] \[D_j = \sum_i T_{ij} \] and the balancing factors are

\[A_i = \frac{1}{\sum_j B_j D_j d_{ij}^{-\beta}}\]
\[B_j = \frac{1}{\sum_i A_i O_i d_{ij}^{-\beta}}\] where 

\begin{itemize}
  \item $T_{ij}$ represents the interaction of flows between the location $i$ (origin) and the location $j$ (destination) within the system. 
  \item $D_{j}$ is a constrained vector equivalent of all flows going to every $j$ destination $j$ from origin $i$ in the system.
  \item $O_{i}$ is a constrained vector equivalent to the total flows from origin $i$ to destinations $j$ in the system.
    \item $d$ is a matrix that contains the distance of cost of interaction between every origin $i$ and destination $j$ of the system.
    \item $B_{j}$, $A_{i}$ and $\beta$ are parameters of the model that need to be estimated.
\end{itemize}


\subsubsection{Parameter calibration}

As the purpose of the next section is to assess the impact of a decrease of employment in the surrounding of a specific underground station, the more appropriate spatial interaction model is the attraction constrained model. The future scenario states that the employment would decrease by 50\% in Canary Wharf, so the model that is sensitive to variations in the attractiveness of the destinations, in this case for employment reasons, is the attraction-constrained model.

After reviewing that the flows and distance data seems to follow a Poisson distribution, a Poisson distribution is carried out to identify the value of the  $\beta$ parameter, which value is equal to: 9.758214913376582e-05.

\subsection{Scenarios}



\subsubsection{Scenario A}

As instructed, the jobs positions in the surrounding area of Cannary Wharf has decreased by 50\%. A new attribute in the data frame is created to store the scenario value of jobs for the entries that have Cannary Wharf as destination station. 

\subsubsection{Scenario B}

\subsubsection{Scenarios discussion}

% Simply use the section and subsection commands, as in this example document! With Overleaf, all the formatting and numbering is handled automatically according to the template you've chosen. If you're using Rich Text mode, you can also create new section and subsections via the buttons in the editor toolbar.

% \subsection{How to include Figures}

% First you have to upload the image file from your computer using the upload link in the file-tree menu. Then use the includegraphics command to include it in your document. Use the figure environment and the caption command to add a number and a caption to your figure. See the code for Figure \ref{fig:frog} in this section for an example.

% Note that your figure will automatically be placed in the most appropriate place for it, given the surrounding text and taking into account other figures or tables that may be close by. You can find out more about adding images to your documents in this help article on \href{https://www.overleaf.com/learn/how-to/Including_images_on_Overleaf}{including images on Overleaf}.

% \begin{figure}
% \centering
% \includegraphics[width=0.3\textwidth]{frog.jpg}
% \caption{\label{fig:frog}This frog was uploaded via the file-tree menu.}
% \end{figure}

% \subsection{How to add Tables}

% Use the table and tabular environments for basic tables --- see Table~\ref{tab:widgets}, for example. For more information, please see this help article on \href{https://www.overleaf.com/learn/latex/tables}{tables}. 

% \begin{table}
% \centering
% \begin{tabular}{l|r}
% Item & Quantity \\\hline
% Widgets & 42 \\
% Gadgets & 13
% \end{tabular}
% \caption{\label{tab:widgets}An example table.}
% \end{table}

% \subsection{How to add Comments and Track Changes}

% Comments can be added to your project by highlighting some text and clicking ``Add comment'' in the top right of the editor pane. To view existing comments, click on the Review menu in the toolbar above. To reply to a comment, click on the Reply button in the lower right corner of the comment. You can close the Review pane by clicking its name on the toolbar when you're done reviewing for the time being.

% Track changes are available on all our \href{https://www.overleaf.com/user/subscription/plans}{premium plans}, and can be toggled on or off using the option at the top of the Review pane. Track changes allow you to keep track of every change made to the document, along with the person making the change. 

% \subsection{How to add Lists}

% You can make lists with automatic numbering \dots

% \begin{enumerate}
% \item Like this,
% \item and like this.
% \end{enumerate}
% \dots or bullet points \dots
% \begin{itemize}
% \item Like this,
% \item and like this.
% \end{itemize}

% \subsection{How to write Mathematics}

% \LaTeX{} is great at typesetting mathematics. Let $X_1, X_2, \ldots, X_n$ be a sequence of independent and identically distributed random variables with $\text{E}[X_i] = \mu$ and $\text{Var}[X_i] = \sigma^2 < \infty$, and let
% \[S_n = \frac{X_1 + X_2 + \cdots + X_n}{n}
%       = \frac{1}{n}\sum_{i}^{n} X_i\]
% denote their mean. Then as $n$ approaches infinity, the random variables $\sqrt{n}(S_n - \mu)$ converge in distribution to a normal $\mathcal{N}(0, \sigma^2)$.


% \subsection{How to change the margins and paper size}

% Usually the template you're using will have the page margins and paper size set correctly for that use-case. For example, if you're using a journal article template provided by the journal publisher, that template will be formatted according to their requirements. In these cases, it's best not to alter the margins directly.

% If however you're using a more general template, such as this one, and would like to alter the margins, a common way to do so is via the geometry package. You can find the geometry package loaded in the preamble at the top of this example file, and if you'd like to learn more about how to adjust the settings, please visit this help article on \href{https://www.overleaf.com/learn/latex/page_size_and_margins}{page size and margins}.

% \subsection{How to change the document language and spell check settings}

% Overleaf supports many different languages, including multiple different languages within one document. 

% To configure the document language, simply edit the option provided to the babel package in the preamble at the top of this example project. To learn more about the different options, please visit this help article on \href{https://www.overleaf.com/learn/latex/International_language_support}{international language support}.

% To change the spell check language, simply open the Overleaf menu at the top left of the editor window, scroll down to the spell check setting, and adjust accordingly.

% \subsection{How to add Citations and a References List}

% You can simply upload a \verb|.bib| file containing your BibTeX entries, created with a tool such as JabRef. You can then cite entries from it, like this: \cite{arcaute_networks_2023}. Just remember to specify a bibliography style, as well as the filename of the \verb|.bib|. You can find a \href{https://www.overleaf.com/help/97-how-to-include-a-bibliography-using-bibtex}{video tutorial here} to learn more about BibTeX.

% If you have an \href{https://www.overleaf.com/user/subscription/plans}{upgraded account}, you can also import your Mendeley or Zotero library directly as a \verb|.bib| file, via the upload menu in the file-tree.

% \subsection{Good luck!}

% We hope you find Overleaf useful, and do take a look at our \href{https://www.overleaf.com/learn}{help library} for more tutorials and user guides! Please also let us know if you have any feedback using the Contact Us link at the bottom of the Overleaf menu --- or use the contact form at \url{https://www.overleaf.com/contact}.

% \bibliographystyle{abbrvnat}
% \setcitestyle{authoryear,open={(},close={)}}
\bibliography{my_bio}

\end{document}